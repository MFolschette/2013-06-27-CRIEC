% What for

\begin{frame}
  \frametitle{What for?}

\begin{center}
\LARGE
  Very well, but…

\Huge
  \vspace*{1em}
  \pause
  \tval{What's the point?}
  \vspace*{.2em}
\end{center}

\pause
\begin{columns}
\begin{column}{.5\textwidth}

\begin{itemize}
  \item Validating the models
  \item Predicting behaviours
  \item Finding gene therapies
\end{itemize}

\end{column}
\end{columns}

\end{frame}



\tikzstyle{abox}=[rounded corners, very thick, align=center]
\tikzstyle{anedge}=[draw=black, ->, shorten <=5pt, shorten >=5pt, very thick]
\tikzstyle{bigbox}=[rounded corners, line width=3pt]
\definecolor{couleurcellule}{rgb}{1,.5,.3}

\begin{frame}
  \frametitle{Experiments \textit{in silico}}

\begin{center}
\scalebox{1.3}{
\begin{tikzpicture}
  \path[use as bounding box] (-1,-3.5) rectangle (5,1);
  % Rectangles labos
  \path[bigbox, draw=blue!50, fill=blue!10] (-2,-1.2) rectangle (6,1.2);
  \node[darkblue!50, anchor=north east] at (6,1.2) {\textsc{wet lab}};
  \path[bigbox, draw=yellow!80, fill=yellow!20] (-2,-1.9) rectangle (6,-4.1);
  \node[darkyellow!50, anchor=north east] at (6,-1.9) {\textsc{dry lab}};
  % Nœuds
  \node[ellipse, draw=pink!100, fill=pink!30, decoration={coil}, decorate, align=center, very thick, inner sep=6.7pt] (sb) {Biological\\system};
  \node[ellipse, draw=violet!80, fill=violet!15, align=center, thick, inner sep=2pt] at (sb) (system) {Biological\\system};
  \node<2->[abox, draw=darkblue, fill=darkblue!30, right of=system, node distance=4cm] (exp) {Experiments\\\textit{in vivo / ex vivo}};
  \node<3-4>[abox, ellipse, draw=notsodarkred, fill=red!30, below of=exp, node distance=3cm] (model) {Model};
  \node<5->[abox, ellipse, draw=notsodarkgreen, fill=green!30, below of=exp, node distance=3cm] (model) {Model};
  \node<4-5>[abox, draw=darkyellow, fill=yellow!40, left of=model, node distance=4cm] (valid) {Validation\\\textit{in silico}};
  \node<7->[abox, draw=darkyellow, fill=yellow!40, draw=darkblue, fill=darkblue!30] at (valid) (predict) {Experiments\\\textit{in silico}};
  % Arcs
  \path<2->[anedge, bend left, shorten <=8pt, shorten >=8pt] (sb) edge (exp);
  \path<3->[anedge] (exp) edge (model);
  \path<4-5,7->[anedge, bend left] (model) edge (valid);
  \path<5>[anedge, bend left] (valid) edge (model.north west);
  \path<5>[anedge, bend right, in=240] (exp) edge (model.north west);
  \path<8>[anedge] (predict) edge (exp.south west);
  \path<8>[anedge, bend left, in=110] (model) edge (exp.south west);
\end{tikzpicture}}
\end{center}
\end{frame}



\tikzstyle{active}=[fill=yellow!70, thick]
\tikzstyle{inactive}=[draw=gray, thick]
\tikzstyle{badgene}=[active, draw=notsodarkred, -, decorate,decoration={zigzag,segment length=1pt,amplitude=.5pt}]
\tikzstyle{inactivebadgene}=[inactive, -, decorate,decoration={zigzag,segment length=1pt,amplitude=.5pt}]

\begin{frame}
  \frametitle{Gene therapies}

\tval{Modify} DNA to cure a disease

\begin{itemize}
  \item Replace a mutated gene \f remove a \tval{harmful protein}
  \item Add a new gene \f produce a \tval{therapeutic protein}
\end{itemize}

\medskip
\begin{center}
\scalebox{1}{
\begin{tikzpicture}[adn]
  %\path[use as bounding box] (-0.7,-0.4) rectangle (2.5,2);
  \node[active] (a) at (-1,0) {a}; \node[active] (b) at (0,0) {b}; \node[active] (c) at (1,0) {c};
  \node[active] (d) at (-.5,-1) {}; \node[active] (e) at (.5,-1) {}; \node[active] (f) at (1.5,-1) {};
  \node[active] (g) at (-1,-2) {}; \node<1>[active] (h) at (0,-2) {n};
  \node[active] (i) at (-1.5,-3) {}; \node<-2>[active] (j) at (-.5,-3) {}; \node<-2>[active] (k) at (.5,-3) {};
  \node<-2>[badgene] (x) at (-1,-4) {x}; \node<-2>[badgene] (y) at (0,-4) {y}; \node<-2>[badgene] (z) at (1,-4) {z};

  \node<2->[inactive] at (h) {n};
  \node<3->[inactive] at (j) {}; \node<3->[inactive] at (k) {};
  \node<3->[inactivebadgene] at (x) {x}; \node<3->[inactivebadgene] at (y) {y}; \node<3->[inactivebadgene] at (z) {z};

%  \uncover<2->{\path[draw=cyan, fill=gray!50, -, thick] (.4,-2) -- (1.5,-2) -- (1.2,-1.8) -- (2.5,-1.8) -- (2.9,-2) -- (1.7,-2) -- (2,-2.2) -- (.4,-2);}
  \uncover<2->{\path[shading=1, left color=notsodarkred, right color=red!30]
    (.4,-2) -- (1.5,-2.1) -- (1.2,-1.9) -- (2.5,-2.05) -- (2.8,-2.25) -- (1.7,-2.1) -- (2,-2.3) -- (.4,-2);}

  \path
    (a) edge[act] (d) (b) edge[act] (d) (b) edge[inh] (e) (c) edge[act] (e) (c) edge[act] (f)
    (d) edge[inh, bend left] (g) (d) edge[inh] (h) (e) edge[act] (h)
    (g) edge[act, bend left] (i) (g) edge[act, bend left] (d) (h) edge[act] (k) (h) edge[act, bend left] (j)
    (i) edge[act, bend left] (g) (j) edge[act] (x) (j) edge[act] (y) (k) edge[act] (z) (j) edge[inh, bend left] (h);
\end{tikzpicture}}
\end{center}
\end{frame}



\begin{frame}[t]
  \frametitle{Back to the Over-approximation}

\begin{columns}[t]
\begin{column}{0.48\textwidth}
\begin{center}
\scalebox{0.55}{
\begin{tikzpicture}
  \path[use as bounding box] (-1,-2.5) rectangle (7,2.5);
  \exatt
  \TState{1-}{a_1,b_1,c_1,d_0}
  \node[objective] (d_2) at (d_2.center) {?};
\end{tikzpicture}
}
\end{center}
\bigskip

\end{column}
\begin{column}{0.52\textwidth}

\tval{Necessary condition}:

\smallskip
There exists a traversal
with no cycle

\smallskip
\begin{itemize}
  \item objective $\rightarrow$ follow \tval{one} solution
  \item solution $\rightarrow$ follow \tval{all} processes
  \item process $\rightarrow$ follow \tval{all} objectives
\end{itemize}
\begin{center}\Large
  \only<2-3>{\textcolor{darkyellow}{\textbf{$R$ is \textbf{true} $\Rightarrow$ Inconclusive}}}%
  \only<5->{\textcolor{red}{$Q$ is \textbf{false} $\Rightarrow$ $R$ is \textbf{false}}}%
\end{center}

\end{column}
\end{columns}

%\bigskip

\begin{center}
\scalebox{\scaleex}{
  \saoinconc
}
\end{center}
\end{frame}
