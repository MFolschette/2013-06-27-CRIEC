% What do I do?

%\newcommand{\centr}[1]{\begin{center}#1\end{center}}
%\newcommand{\spacr}[1]{\Large \vspace{.0em}#1\vspace{.0em}}
%\newcommand{\spacr}[1]{\vspace{-1em}\centr{\Large #1}}

\newcommand{\centr}[1]{#1\vspace*{.5em}}
\newcommand{\spacr}[1]{{\Large #1}}

\begin{frame}
  \frametitle{What do I do?}

\f Problem: easy to understand but hard to study (\tval{exponential})

\bigskip
\begin{tabular}{cc}%>{\vspace*{.5em}}
  \tval{Model} & \tval{Possible configs} \\\hline
  \tikz[adn]{\path[use as bounding box] (-.2,-.1) rectangle (1.7,.5);
             \node(a){a};\node[right of=a](b){b};
             \path(a)edge[bend left,act](b) (b)edge[bend left,inh](a);}
    & \centr{\ex{4}}\\\pause
  \tikz[adn]{\path[use as bounding box] (-1.7,-.1) rectangle (1.7,.5);
             \node(a){a};\node[right of=a](b){b};\node[left of=a](c){c};
             \path(a)edge[bend left,act](b) (b)edge[bend left,inh](a) (c)edge[act](a);}
    & \vspace{-.5em}\centr{\ex{8}}\vspace{.5em}\\\pause
  \centr{$\vdots$}
    & \centr{$\vdots$} \\
  \spacr{(10)}
    & \centr{\ex{1024}}\\\pause
  \spacr{(20)}
    & \centr{\ex{1048576}}\\\pause
  \spacr{(100)}
    & \centr{\ex{1267650600000000000000000000000}}\\
\end{tabular}

\end{frame}


\subsection{The Process Hitting framework}

\begin{frame}[t]
  \frametitle{The Process Hitting modelling}
  \framesubtitle{\tcite{\citepmrtcsb}}

% 1 : Sortes
\only<1>{
\tikzstyle{process}=[circle,minimum size=15pt,font=\footnotesize,inner sep=1pt]
\tikzstyle{tick label}=[color=white, font=\footnotesize]
\tikzstyle{tick}=[transparent]
\tikzstyle{hit}=[transparent]
\tikzstyle{selfhit}=[transparent, min distance=30pt,curve to]
\tikzstyle{bounce}=[transparent]
\tikzstyle{hlhit}=[transparent]
\begin{center}\scalebox{\scaleex}{
\begin{tikzpicture}
  \exphdef
\end{tikzpicture}
}\end{center}
}

% 2 : Processus
\only<2>{
\tikzstyle{process}=[circle,draw,minimum size=15pt,font=\footnotesize,inner sep=1pt]
\tikzstyle{tick label}=[font=\footnotesize]
\tikzstyle{tick}=[densely dotted]
\tikzstyle{hit}=[transparent]
\tikzstyle{selfhit}=[transparent, min distance=30pt,curve to]
\tikzstyle{bounce}=[transparent]
\tikzstyle{hlhit}=[transparent]
\begin{center}\scalebox{\scaleex}{
\begin{tikzpicture}
  \exphdef
\end{tikzpicture}
}\end{center}
}

% 3 : États
\only<3>{
\tikzstyle{hit}=[transparent]
\tikzstyle{selfhit}=[transparent, min distance=30pt,curve to]
\tikzstyle{bounce}=[transparent]
\tikzstyle{hlhit}=[transparent]
\begin{center}\scalebox{\scaleex}{
\begin{tikzpicture}
  \exphdef

  \TState{3}{a_0,b_1,z_0}
\end{tikzpicture}
}\end{center}
}

% 4 : Actions
\only<4->{
\tikzstyle{tick}=[densely dotted]
\tikzstyle{hit}=[->,>=angle 45]
\tikzstyle{selfhit}=[min distance=30pt,curve to]
\tikzstyle{bounce}=[densely dotted,>=stealth',->]
\tikzstyle{hlhit}=[very thick]
\begin{center}\scalebox{\scaleex}{
\begin{tikzpicture}
  \exphdef

  \TState{4-5}{a_0,b_1,z_0}
  \TState{6}{a_0,b_1,z_1}
  \TState{7}{a_1,b_1,z_1}
  \TState{8}{a_1,b_1,z_2}

  \only<5>{
    \THit{b_1}{hl}{z_0}{.west}{z_1}
    \path[bounce,bend left,hl] \TBounce{z_0}{}{z_1}{.south};
  }
  \only<6>{
    \THit{a_0}{out=250,in=200,selfhit,hl}{a_0}{.west}{a_1}
    \path[bounce,bend left,hl] \TBounce{a_0}{}{a_1}{.south};
  }
  \only<7>{
    \THit{a_1}{hl}{z_1}{.west}{z_2}
    \path[bounce,bend left,hl] \TBounce{z_1}{}{z_2}{.south};
  }
\end{tikzpicture}
}\end{center}
}

%\medskip
\begin{liste}
  \item \tval{Sorts}: components \qex{$a$, $b$, $z$}
\pause[2]
  \item \tval{Processes}: local states / levels of expression \qex{$z_0$, $z_1$, $z_2$}
\pause[3]
  \item \tval{States}: sets of active processes%
  \only<3-5>{\qex{$\PHetat{a_0, b_1, z_0}$}}%
  \only<6>{\qex{$\PHetat{a_0, b_1, z_1}$}}%
  \only<7>{\qex{$\PHetat{a_1, b_1, z_1}$}}%
  \only<8>{\qex{$\PHetat{a_1, b_1, z_2}$}}%
\pause[4]
  \item \tval{Actions}: dynamics \qex{\only<5>{\underline}{$\PHfrappe{b_1}{z_0}{z_1}$}, \only<6>{\underline}{$\PHfrappe{a_0}{a_0}{a_1}$}, \only<7>{\underline}{$\PHfrappe{a_1}{z_1}{z_2}$}}
\end{liste}
\end{frame}



\subsection{Static analysis}

\newcommand{\sortname}[1]{\tikz{\path[use as bounding box] (-.2,-.1) rectangle (.2,.12);\node[sort] at (0,0) {#1};}}

\begin{frame}[t]
  \frametitle{Static analysis}
  \framesubtitle{\tcite{\citepmrmscs}}

{
\tikzstyle{tick}=[densely dotted]
\tikzstyle{hit}=[->,>=angle 45]
\tikzstyle{selfhit}=[min distance=30pt,curve to]
\tikzstyle{bounce}=[densely dotted,>=stealth',->]
\tikzstyle{hlhit}=[very thick]
\begin{center}\scalebox{\scaleex}{
\begin{tikzpicture}
  \exphdef

  \TState{1-}{a_0,b_1,z_0}
  \node[process,very thick] at (z_2.center) {?};
\end{tikzpicture}
}\end{center}
}

Questions:
\begin{itemize}
  \item Can I \tval{reach} level $2$ of \sortname{z}?
  \item Do I need to \tval{knock-out} \sortname{b}, what changes?
  \item If I add \tval{new actions} on \sortname{a}, what changes?
\end{itemize}
\end{frame}



\begin{frame}
  \frametitle{Over- and Under-approximations}
  \framesubtitle{\tcite{\citepmrmscs}}

\begin{fleches}
  \item Directly checking $R$ is hard (\tval{exponential})
  \item Rather check \tval{approximations} $P$ and $Q$:
\end{fleches}

\begin{center}
\scalebox{0.7}{
  \figsa
}
\end{center}

\uncover<8->{
%Polynomial w.r.t.~the number of sorts and \\exponential w.r.t.~the number of processes in each sort
Computing $P$ or $Q$ is much simpler (roughly \tval{polynomial})
\begin{fleches}
  \item Efficient for big models \f \tval{Hundredths of seconds}
\end{fleches}
}
\end{frame}
