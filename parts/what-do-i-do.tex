% What do I do?

%\newcommand{\centr}[1]{\begin{center}#1\end{center}}
%\newcommand{\spacr}[1]{\Large \vspace{.0em}#1\vspace{.0em}}
%\newcommand{\spacr}[1]{\vspace{-1em}\centr{\Large #1}}

\newcommand{\centr}[1]{#1\vspace*{.5em}}
\newcommand{\spacr}[1]{{\Large #1}}

\begin{frame}
  \frametitle{What do I do?}

\f Problem: easy to understand but hard to study (\tval{exponential})

\bigskip
\begin{tabular}{cc}%>{\vspace*{.5em}}
  \tval{Model} & \tval{Possible configs} \\\hline
  \tikz[adn]{\path[use as bounding box] (-.2,-.1) rectangle (1.7,.5);
             \node(a){a};\node[right of=a](b){b};
             \path(a)edge[bend left,act](b) (b)edge[bend left,inh](a);}
    & \centr{\ex{4}}\\\pause
  \tikz[adn]{\path[use as bounding box] (-1.7,-.1) rectangle (1.7,.5);
             \node(a){a};\node[right of=a](b){b};\node[left of=a](c){c};
             \path(a)edge[bend left,act](b) (b)edge[bend left,inh](a) (c)edge[act](a);}
    & \vspace{-.5em}\centr{\ex{8}}\vspace{.5em}\\\pause
  \centr{$\vdots$}
    & \centr{$\vdots$} \\
  \spacr{(10)}
    & \centr{\ex{1024}}\\\pause
  \spacr{(20)}
    & \centr{\ex{1048576}}\\\pause
  \spacr{(100)}
    & \centr{\ex{1267650600000000000000000000000}}\\
\end{tabular}

\end{frame}


\subsection{The Process Hitting framework}

\begin{frame}[t]
  \frametitle{The Process Hitting modelling}
  \framesubtitle{\tcite{\citepmrtcsb}}

% 1 : Sortes
\only<1>{
\tikzstyle{process}=[circle,minimum size=15pt,font=\footnotesize,inner sep=1pt]
\tikzstyle{tick label}=[color=white, font=\footnotesize]
\tikzstyle{tick}=[transparent]
\tikzstyle{hit}=[transparent]
\tikzstyle{selfhit}=[transparent, min distance=30pt,curve to]
\tikzstyle{bounce}=[transparent]
\tikzstyle{hlhit}=[transparent]
\begin{center}\scalebox{\scaleex}{
\begin{tikzpicture}
  \exphdef
\end{tikzpicture}
}\end{center}
}

% 2 : Processus
\only<2>{
\tikzstyle{process}=[circle,draw,minimum size=15pt,font=\footnotesize,inner sep=1pt]
\tikzstyle{tick label}=[font=\footnotesize]
\tikzstyle{tick}=[densely dotted]
\tikzstyle{hit}=[transparent]
\tikzstyle{selfhit}=[transparent, min distance=30pt,curve to]
\tikzstyle{bounce}=[transparent]
\tikzstyle{hlhit}=[transparent]
\begin{center}\scalebox{\scaleex}{
\begin{tikzpicture}
  \exphdef
\end{tikzpicture}
}\end{center}
}

% 3 : États
\only<3>{
\tikzstyle{hit}=[transparent]
\tikzstyle{selfhit}=[transparent, min distance=30pt,curve to]
\tikzstyle{bounce}=[transparent]
\tikzstyle{hlhit}=[transparent]
\begin{center}\scalebox{\scaleex}{
\begin{tikzpicture}
  \exphdef

  \TState{3}{a_0,b_1,z_0}
\end{tikzpicture}
}\end{center}
}

% 4 : Actions
\only<4->{
\tikzstyle{tick}=[densely dotted]
\tikzstyle{hit}=[->,>=angle 45]
\tikzstyle{selfhit}=[min distance=30pt,curve to]
\tikzstyle{bounce}=[densely dotted,>=stealth',->]
\tikzstyle{hlhit}=[very thick]
\begin{center}\scalebox{\scaleex}{
\begin{tikzpicture}
  \exphdef

  \TState{4-5}{a_0,b_1,z_0}
  \TState{6}{a_0,b_1,z_1}
  \TState{7}{a_1,b_1,z_1}
  \TState{8}{a_1,b_1,z_2}

  \only<5>{
    \THit{b_1}{hl}{z_0}{.west}{z_1}
    \path[bounce,bend left,hl] \TBounce{z_0}{}{z_1}{.south};
  }
  \only<6>{
    \THit{a_0}{out=250,in=200,selfhit,hl}{a_0}{.west}{a_1}
    \path[bounce,bend left,hl] \TBounce{a_0}{}{a_1}{.south};
  }
  \only<7>{
    \THit{a_1}{hl}{z_1}{.west}{z_2}
    \path[bounce,bend left,hl] \TBounce{z_1}{}{z_2}{.south};
  }
\end{tikzpicture}
}\end{center}
}

%\medskip
\begin{liste}
  \item \tval{Sorts}: components \qex{$a$, $b$, $z$}
\pause[2]
  \item \tval{Processes}: local states / levels of expression \qex{$z_0$, $z_1$, $z_2$}
\pause[3]
  \item \tval{States}: sets of active processes%
  \only<3-5>{\qex{$\PHetat{a_0, b_1, z_0}$}}%
  \only<6>{\qex{$\PHetat{a_0, b_1, z_1}$}}%
  \only<7>{\qex{$\PHetat{a_1, b_1, z_1}$}}%
  \only<8>{\qex{$\PHetat{a_1, b_1, z_2}$}}%
\pause[4]
  \item \tval{Actions}: dynamics \qex{\only<5>{\underline}{$\PHfrappe{b_1}{z_0}{z_1}$}, \only<6>{\underline}{$\PHfrappe{a_0}{a_0}{a_1}$}, \only<7>{\underline}{$\PHfrappe{a_1}{z_1}{z_2}$}}
\end{liste}
\end{frame}



\subsection{Static analysis}

\begin{frame}[t]
  \frametitle{Static analysis: successive reachability}
  \framesubtitle{\tcite{\citepmrmscs}}

%Successive reachability of processes:

\begin{columns}
\begin{column}{0.55\textwidth}

\begin{center}
\scalebox{0.75}{
\begin{tikzpicture}
  \path[use as bounding box] (-1,-3) rectangle (7,2.7);
  \exatt

  \TState{1-4}{a_0,b_0,c_0,d_0}

  \TState{5}{a_0,b_0,c_0,d_0}
  \TState{6}{a_0,b_0,c_1,d_0}
  \TState{7}{a_0,b_0,c_1,d_1}
  \TState{8}{a_0,b_1,c_1,d_1}
  \TState{9}{a_0,b_1,c_1,d_2}

  \node<2>[objective] at (d_1.center) {1?};
  \node<2>[objective] at (d_2.center) {2?};

  \node<3>[objective] at (d_1.center) {1?};
  \node<3>[objective] at (b_1.center) {2?};
  \node<3>[objective] at (d_2.center) {3?};

  \node<4-8>[objective] at (d_2.center) {1?};
  %\node<9>[process,very thick,fill=none] at (d_2.center) {};

  \only<5>{\THit{a_0}{hlhit}{c_0}{.north}{c_1}}
  \path<5>[bounce,bend left,hlhit] \TBounce{c_0}{}{c_1}{.west};
  \only<6>{\THit{b_0}{hlhit}{d_0}{.west}{d_1}}
  \path<6>[bounce,bend left,hlhit] \TBounce{d_0}{}{d_1}{.south};
  \only<7>{\THit{c_1}{bend left=20pt,hlhit}{b_0}{.west}{b_1}}
  \path<7>[bounce,bend left,hlhit] \TBounce{b_0}{}{b_1}{.south};
  \only<8>{\THit{b_1}{hlhit}{d_1}{.west}{d_2}}
  \path<8>[bounce,bend left,hlhit] \TBounce{d_1}{}{d_2}{.south};
\end{tikzpicture}
}
\end{center}

\end{column}
\begin{column}{0.45\textwidth}

%\pause
~\\~\\
\begin{itemize}
  \item Initial state
    \\ \rex{\PHetat{a_1, b_0, c_0, d_0}} \pause
  \item Objectives
    \\ \rex{$[\ \Rsh d_1\ \PHconcat\ \Rsh d_2\ ]$} \pause
    \\\smallskip \rex{$[\ \Rsh d_1 \PHconcat\ \Rsh b_1 \PHconcat\ \Rsh d_2\ ]$} \pause
    \\\smallskip \rex{$[\ \Rsh d_2\ ]$} \pause
\end{itemize}

\end{column}
\end{columns}

\medskip
\begin{center}
\f Concretization of the objective = scenario

\ex{
$ \only<5>{\underline{\PHfrappe{a_0}{c_0}{c_1}}} \only<-4,6->{\PHfrappe{a_0}{c_0}{c_1}} \PHconcat %
  \only<6>{\underline{\PHfrappe{b_0}{d_0}{d_1}}}\only<-5,7->{\PHfrappe{b_0}{d_0}{d_1}} \PHconcat %
  \only<7>{\underline{\PHfrappe{c_1}{b_0}{b_1}}}\only<-6,8->{\PHfrappe{c_1}{b_0}{b_1}} \PHconcat %
  \only<8>{\underline{\PHfrappe{b_1}{d_1}{d_2}}}\only<-7,9->{\PHfrappe{b_1}{d_1}{d_2}}
$}
\end{center}
\end{frame}



\begin{frame}
  \frametitle{Over- and Under-approximations}
  \framesubtitle{\tcite{\citepmrmscs}}

\begin{fleches}
  \item Directly checking $R$ is hard (\tval{exponential})
  \item Rather check \tval{approximations} $P$ and $Q$ so that: $\underline{P \Rightarrow R \Rightarrow Q}$
\end{fleches}

\begin{center}
\scalebox{0.7}{
  \figsa
}
\end{center}

\uncover<8->{
%Polynomial w.r.t.~the number of sorts and \\exponential w.r.t.~the number of processes in each sort
Computing $P$ or $Q$ is much simpler (roughly \tval{polynomial})
\begin{fleches}
  \item Efficient for big models \f \tval{Hundredths of seconds}
\end{fleches}
}
\end{frame}



\begin{frame}[t]
  \frametitle{Under-approximation}

\def \tu {3}
\def \tub {4}
\def \tuf {5}

\begin{columns}[t]
\begin{column}{0.48\textwidth}
\begin{center}
\scalebox{0.55}{
\begin{tikzpicture}
  \path[use as bounding box] (-1,-2.7) rectangle (7,1.5);
  \exatt
  \TState{-\tu}{a_1,b_1,c_1,d_0}
  \TState{\tub-}{a_0,b_1,c_0,d_0}
  \node[objective] (d_2) at (d_2.center) {?};
\end{tikzpicture}
}
\end{center}

\end{column}
\begin{column}{0.52\textwidth}

\uncover<2->{
%\vspace{1em}
\tval{Sufficient condition}:

\smallskip
\begin{itemize}
  \item no cycle
  \item \only<-\tu>{each objective has a solution} \only<\tub->{\sout{each objective has a solution}}
\end{itemize}
\begin{center}\Large%
  \only<\tu>{\textcolor{darkgreen}{$P$ is \textbf{true} $\Rightarrow$ $R$ is \textbf{true}}}%
  \only<\tuf>{\textcolor{darkyellow}{$P$ is \textbf{false} $\Rightarrow$ \textbf{Inconclusive}}}
\end{center}
}

\end{column}
\end{columns}

\begin{center}%
%\vspace*{1cm}%
\scalebox{\scaleex}{%
\only<-\tu>{%
\scalebox{\scaleex}{%
\begin{tikzpicture}[aS]
  \path[use as bounding box] (.7,1) rectangle (5.8,2.5);

  \glclegend{}{$d_2$}{$\PHobj{d_0}{d_2}$}
\end{tikzpicture}
}
~~~\sauyes
}
\only<\tub->{
  \sauinconc
}}
\end{center}
\end{frame}



\begin{frame}[t]
  \frametitle{Over-approximation}

\def \to {4}
\def \tob {5}
\def \tof {6}
\def \tokp {7}

\begin{columns}[t]
\begin{column}{0.48\textwidth}
\begin{center}
\scalebox{0.55}{
\begin{tikzpicture}
  \path[use as bounding box] (-1,-3.5) rectangle (7,1.5);
  \exatt
  \TState{-\to}{a_1,b_0,c_0,d_1}
  \TState{\tob-}{a_1,b_1,c_1,d_0}
  \node[objective] (d_2) at (d_2.center) {?};
\end{tikzpicture}
}
\end{center}
\bigskip

\end{column}
\begin{column}{0.52\textwidth}

\tval{Necessary condition}:

\smallskip
\only<2->{
\only<3-\to>{\sout{There exists a traversal}}\only<2,\tob->{There exists a traversal}
with no cycle

\smallskip
\begin{itemize}
  \item \only<3-\to>{\sout{objective $\rightarrow$ follow \tval{one} solution}}\only<1-2,\tob->{objective $\rightarrow$ follow \tval{one} solution}%
  \item solution $\rightarrow$ follow \tval{all} processes
  \item process $\rightarrow$ follow \tval{all} objectives
\end{itemize}
\begin{center}\Large%
  \only<\to>{\textcolor{red}{$Q$ is \textbf{false} $\Rightarrow$ $R$ is \textbf{false}}}%
  \only<\tof->{\textcolor{darkyellow}{\textbf{$R$ is \textbf{true} $\Rightarrow$ Inconclusive}}}
\end{center}
}

\end{column}
\end{columns}

%\bigskip

\begin{center}
\scalebox{\scaleex}{
\only<1-\to>{
  \saono
}
\only<\tob->{
  \saoinconc
}}
\end{center}
\end{frame}
