\documentclass[fleqn,10pt,c]{beamer}

\usepackage[english]{babel}
\usepackage[utf8]{inputenc}
\usepackage[T1]{fontenc}
%\usepackage{french} % Sommaire en début de document
%\usepackage[top=2cm, bottom=2cm, left=2cm, right=2cm]{geometry} % Marges

\usepackage{amsmath} % Maths
\usepackage{amsfonts} % Maths
\usepackage{amssymb} % Maths
\usepackage{stmaryrd} % Maths (crochets doubles)

%\usepackage{listings} % Mise en forme du code (pour Hoare) ## À REVOIR ###
%\usepackage{ifthen} % Structures If Then Else
\usepackage{theorem} % Styles supplémentaires pour théorèmes
\usepackage{url}
\usepackage{array}  % Tableaux évolués
\usepackage{multirow}  % Pour des colonnes sur plusieurs lignes

%\usepackage{enumerate} % Changer les puces des listes d'énumération
%\usepackage{setspace} % Changer les interlignes

%\usepackage{subfig} % Créer des sous-figures
%\usepackage{graphicx} % Importer des images

\usepackage{ulem}  % Pour l'attribut barré

\usepackage{comment}

% Police
\usepackage{lmodern}
%\usepackage{libertine}


\usepackage{tikz}
\usetikzlibrary[decorations.pathmorphing]

\input{macros/macros}
\input{macros/macros-ph}
\input{macros/tikzstyles.tex}
\input{macros/macros-abstr}


% Commande À FAIRE
\usepackage{color} % Couleurs du texte
\newcommand{\afaire}[1]{\textcolor{red}{[À FAIRE : #1]}}
\newcommand{\todo}[1]{\textcolor{red}{\textbf{[TODO\ifthenelse{\equal{#1}{}}{}{: #1}]}}}



\definecolor{blueblue}{rgb}{0,.4,0.7}
\definecolor{bluebluecyan}{rgb}{0,.6,0.8}

\colorlet{couleurtheme}{blueblue!80}  % Couleur principale du thème
\colorlet{couleurcit}{purple}  % Couleur des citations
\colorlet{couleurex}{blue}  % Couleur des exemples
\colorlet{couleurredex}{red}  % Couleur des exemples
\colorlet{couleurliens}{darkblue}  % Couleur des liens

\usetheme{Pittsburgh}   % Thème général
\usefonttheme{default}  % Thème de polices
\setbeamertemplate{navigation symbols}{}  % Pas de menu de navigation
%\setbeamertemplate{itemize item}[x]   % Puces des listes

\usecolortheme[named=couleurtheme]{structure}    % Couleur de la structure : titres et puces
%\setbeamercolor{normal text}{bg=black,fg=white}  % Couleur du texte
\setbeamercolor{item}{fg=couleurtheme}           % Couleur des puces
%\setbeamercolor{item projected}{fg=black}        % Couleur des recouvrements
%\setbeamercolor{alerted text}{fg=yellow}         % ?

\setbeamerfont{frametitle}{size=\Large}  % Police des titres


% Flèche grise
\newcommand{\f}{\textcolor{couleurtheme}{\textbf{$\rightarrow$\ }}}
\newcommand{\F}{\textcolor{couleurtheme}{\textbf{$\Rightarrow$\ }}}

% Environnement liste avec flèches
\newenvironment{fleches}{%
\begin{list}{}{%
\setlength{\labelwidth}{1em}% largeur de la boîte englobant le label
\setlength{\labelsep}{0pt}% espace entre paragraphe et l’étiquette
%\setlength{\itemsep}{1pt}
%\setlength{\leftmargin}{\labelwidth+\labelsep}% marge de gauche
\renewcommand{\makelabel}{\f}%
}}{\end{list}}

% Liste sans puce
\newenvironment{liste}{%
\begin{list}{}{%
\setlength{\labelwidth}{0em}% largeur de la boîte englobant le label
\setlength{\labelsep}{0pt}% espace entre paragraphe et l’étiquette
\setlength{\leftmargin}{0em}% marge de gauche
%\renewcommand{\makelabel}{\f}%
}}{\end{list}}

% Style des exemples
\newcommand{\ex}[1]{\textcolor{couleurex}{#1}}
\newcommand{\qex}[1]{\quad \ex{#1}}
\newcommand{\rex}[1]{\hfill \ex{#1}}
\newcommand{\redex}[1]{\textcolor{couleurredex}{#1}}

\newcommand{\lien}[1]{\textcolor{couleurliens}{\underline{\url{#1}}}}

%\newcommand{\console}[1]{\textcolor{darkgray}{#1}}

% Style des citations
\newcommand{\tscite}[1]{\textcolor{couleurcit}{#1}}
\newcommand{\tcite}[1]{\tscite{[#1]}}
\newcommand{\tcitebullet}{~~$\textcolor{couleurtheme}{\bullet}$~}



% Style de texte mis en valeur
\newcommand{\tval}[1]{\textbf{#1}}

% Un vrai symbole pour l'ensemble vide
\renewcommand{\emptyset}{\varnothing}

% Pour définir la conférence et son nom court
\newcommand{\conference}[2]{\def\theconference{#2}
\def\insertshortconference{\ifthenelse{\equal{#1}{-}}{#2}{\ifthenelse{\equal{#1}{}}{#2}{#1}}}}



\newcommand{\thedate}{2013/06/27}
\date{\thedate}
\conference{CRIEC'13}{CRIEC 2013}
\title[The Process Hitting framework]{The Process Hitting framework:\\a qualitative Bio-informatics modelling}
\author{Maxime FOLSCHETTE}




\setbeamertemplate{footline}{\color{couleurtheme}%
\scriptsize
\quad\strut%
\insertauthor%
\hfill%
\insertframenumber/\inserttotalframenumber%
\hfill%
\insertshortconference{} --- \thedate\quad\strut
}


\newcommand{\headersep}{$\circ$} % \bullet \triangleright

\setbeamertemplate{headline}{\color{couleurtheme}%
\vskip0.3em%
\quad\strut%
{\scriptsize\color{black}%
% Gris si une section existe
\ifthenelse{\equal{\thesection}{0}}{}{%
\ifthenelse{\equal{\lastsection}{x}}{}{%
\color{couleurtheme}%
}}%
\insertshorttitle
\ifthenelse{\equal{\thesection}{0}}{}{%
\ifthenelse{\equal{\lastsection}{x}}{}{%
~\headersep{} %
% Gris si une sous-section existe
\ifthenelse{\equal{\thesubsection}{0}}{\color{black}}{%
\ifthenelse{\equal{\lastsubsection}{x}}{\color{black}}{%
\color{couleurtheme}%
}}%
\insertsectionhead%
%
\ifthenelse{\equal{\thesubsection}{0}}{}{%
\ifthenelse{\equal{\lastsubsection}{x}}{}{%
~\headersep{} \color{black}\insertsubsectionhead%
%
}}}}}%
\vskip-5ex%
}



\def \scaleex {0.85}
\def \scaleminiex {0.6}
\def \scaleinf {0.6}

\colorlet{colorb}{blue}
\colorlet{colora1}{yellow}
\colorlet{colora0}{green}
\colorlet{colora1font}{darkyellow}
\colorlet{colora0font}{darkgreen}

\colorlet{exanswer}{blue}
\colorlet{colorgray}{lightgray}

%\definecolor{colortitle}{rgb}{0.54,0.8,0.9}
\colorlet{colortitle}{bluebluecyan!50}



\begin{document}

\begin{frame}[plain,label=title]

% Cadre de titre
\begin{center}
\vspace{1cm}
\setbeamercolor{postit}{fg=black,bg=colortitle}
\begin{beamercolorbox}[sep=0.5em]{postit}
\centering
\Large
\textbf{%
{\normalsize\theconference{}}\\~\\%
\inserttitle
}
\end{beamercolorbox}

% Auteurs et instituts
\par
\medskip
\bigskip
\normalsize
Maxime FOLSCHETTE

\medskip
\footnotesize
MeForBio / IRCCyN / École Centrale de Nantes (Nantes, France)

\texttt{maxime.folschette@irccyn.ec-nantes.fr}

\url{http://www.irccyn.ec-nantes.fr/~folschet/}
\end{center}

\end{frame}



\input{parts/ex.tex}

%%% Citations
\newcommand{\citeegfra}{\quad\tval{\ex{egfr20}}: \tcite{Epidermal Growth Factor Receptor, by Özgür Sahin \textit{et al.}}}
\newcommand{\citeegfrb}{\quad\tval{\ex{egfr104}}: \tcite{Epidermal Growth Factor Receptor, by Regina Samaga \textit{et al.}}}
\newcommand{\citetcrsiga}{\quad\tval{\ex{tcrsig40}}: \tcite{T-Cell Receptor Signaling, by Steffen Klamt \textit{et al.}}}
\newcommand{\citetcrsigb}{\quad\tval{\ex{tcrsig94}}: \tcite{T-Cell Receptor Signaling, by Julio Saez-Rodriguez \textit{et al.}}}

\newcommand{\citemodels}{\bigskip\citeegfra\\\citeegfrb\\\citetcrsiga\\\citetcrsigb}

\newcommand{\citepmrtcsb}{Paulevé \textit{et al.}, \textit{Transactions on Computational Systems Biology}, 2011}
\newcommand{\citepmrmscs}{Paulevé \textit{et al.}, \textit{Mathematical Structures in Computer Science}, 2012}
\newcommand{\citefpimrcmsb}{Folschette \textit{et al.}, \textit{Computational Methods in Systems Biology}, 2012}
\newcommand{\citefpmrcstobio}{Folschette \textit{et al.}, \textit{CS2Bio}, 2013}
\newcommand{\citedejong}{De Jong, \textit{Journal of Computational Biology}, 2002}



\section{Introduction}
% Diapo d'intro

\begin{frame}[c]
  \frametitle{Context and Aims}


\begin{center}
\Large
\tval{MeForBio} team:\large\\
Qualitative modelling to study\\
complex dynamical biological systems
\end{center}

\bigskip
\normalsize
\pause
\begin{enumerate}[1)]
  \item What is Bio-informatics?
  \begin{itemize}
    \item[] Studying \tval{gene interactions} with mathematical tools
  \end{itemize}

\pause
  \smallskip
  \item What do I do?
  \begin{itemize}
    \item[] Efficient methods thanks to the \tval{Process Hitting} framework
  \end{itemize}

\pause
  \smallskip
  \item What for?
  \begin{itemize}
    \item[] Understanding leads to \tval{solutions}
  \end{itemize}
\end{enumerate}

\end{frame}


\section{What is Bio-informatics?}
% What is bio-informatics

\newcommand{\lf}{\tikz{\path[use as bounding box] (0,-.1) rectangle (0,.08);\path[draw=couleurtheme,very thick,|->] (0,0) -- (1.2,0);}}

\begin{frame}
  \frametitle{What is Bio-informatics?}

\begin{center}
\Large
  Confluence” of \tval{Biology} and \tval{Computer Science}
\end{center}

%\smallskip
\pause
\tval{Computer Science}: science of processing information

\pause
\medskip
\tval{Biology}: study of living organisms

\pause
\medskip
\begin{columns}
\begin{column}{.37\textwidth}

Many fields:
\begin{itemize}
  \item Sequencing
  \item \only<4>{Gene regulations}\only<5->{\tval{Gene regulations \lf}}
  \item Simulation
  \item Experiments
  \item …
\end{itemize}

~

~

\end{column}
\begin{column}{.55\textwidth}

\uncover<5->{
~

~

~

Approaches:
\begin{itemize}
  \item Differential equations
  \item \only<5>{Algebraic/qualitative}\only<6->{\tval{Algebraic/qualitative}}
  \item Hybrid
  \item Stochastic/probabilistic
  \item …
\end{itemize}
}

\end{column}
\end{columns}
\end{frame}



\begin{frame}
  \frametitle{Gene regulations}

\only<1>{\begin{center}
  \includegraphics[width=.9\textwidth]{figs/protein.png}
\end{center}}
\only<2>{\begin{center}
  \scalebox{10}{
  \begin{tikzpicture}[adn]
    \node (a) {a};
  \end{tikzpicture}}
\end{center}}

\end{frame}



\newcommand{\noeud}[1]{\tikz[adn]{\path[use as bounding box] (-.25,-.1) rectangle (.25,.12);\node[inner sep=0] {#1};}}

\begin{frame}[c]
  \frametitle{Usual biological algebraic models}
  \framesubtitle{\tcite{\citedejong}}

\begin{center}
Modelling interacting genes/proteins:

%\bigskip

\scalebox{1.3}{
\begin{tikzpicture}[adn]
  \path[use as bounding box] (-0.7,-0.4) rectangle (2.5,2);
  \node[inner sep=0] (z) at (2,0.75) {z};
  \node[inner sep=0] (a) at (0,1.5) {a};
  \node[inner sep=0] (b) at (0,0) {b};
%  \path
%    node[elabel, above=-1em of a] {$\segm{0}{1}$}
%    node[elabel, below=-1em of b] {$\segm{0}{1}$}
%    node[elabel, below=-1em of z] {$\segm{0}{2}$};
  \path
    (a) edge[act,bend right=15] node[elabel,left] {$+$} (b)
    (b) edge[act,bend right=15] node[elabel,right] {$+$} (a)
    (b) edge[inh,loop left] node[elabel,left] {$-$} (b)
    (a) edge[inh] node[elabel,above] {$-$} (z)
    (b) edge[act] node[elabel,below] {$+$} (z);
\end{tikzpicture}}
\end{center}

\pause
\begin{columns}
\begin{column}{.6\textwidth}

Questions:
\begin{itemize}
  \item How does \noeud{z} \tval{behave}?
  \item Is it \tval{possible} to make \noeud{a} inactive?
  \item If I \tval{knock-out} \noeud{b}, what changes?
\end{itemize}

\end{column}
\end{columns}

\end{frame}


\section{What do I do?}
% What do I do?

%\newcommand{\centr}[1]{\begin{center}#1\end{center}}
%\newcommand{\spacr}[1]{\Large \vspace{.0em}#1\vspace{.0em}}
%\newcommand{\spacr}[1]{\vspace{-1em}\centr{\Large #1}}

\newcommand{\centr}[1]{#1\vspace*{.5em}}
\newcommand{\spacr}[1]{{\Large #1}}

\begin{frame}
  \frametitle{What do I do?}

\f Problem: easy to understand but hard to study (\tval{exponential})

\bigskip
\begin{tabular}{cc}%>{\vspace*{.5em}}
  \tval{Model} & \tval{Possible configs} \\\hline
  \tikz[adn]{\path[use as bounding box] (-.2,-.1) rectangle (1.7,.5);
             \node(a){a};\node[right of=a](b){b};
             \path(a)edge[bend left,act](b) (b)edge[bend left,inh](a);}
    & \centr{\ex{4}}\\\pause
  \tikz[adn]{\path[use as bounding box] (-1.7,-.1) rectangle (1.7,.5);
             \node(a){a};\node[right of=a](b){b};\node[left of=a](c){c};
             \path(a)edge[bend left,act](b) (b)edge[bend left,inh](a) (c)edge[act](a);}
    & \vspace{-.5em}\centr{\ex{8}}\vspace{.5em}\\\pause
  \centr{$\vdots$}
    & \centr{$\vdots$} \\
  \spacr{(10)}
    & \centr{\ex{1024}}\\\pause
  \spacr{(20)}
    & \centr{\ex{1048576}}\\\pause
  \spacr{(100)}
    & \centr{\ex{1267650600000000000000000000000}}\\
\end{tabular}

\end{frame}


\subsection{The Process Hitting framework}

\begin{frame}[t]
  \frametitle{The Process Hitting modelling}
  \framesubtitle{\tcite{\citepmrtcsb}}

% 1 : Sortes
\only<1>{
\tikzstyle{process}=[circle,minimum size=15pt,font=\footnotesize,inner sep=1pt]
\tikzstyle{tick label}=[color=white, font=\footnotesize]
\tikzstyle{tick}=[transparent]
\tikzstyle{hit}=[transparent]
\tikzstyle{selfhit}=[transparent, min distance=30pt,curve to]
\tikzstyle{bounce}=[transparent]
\tikzstyle{hlhit}=[transparent]
\begin{center}\scalebox{\scaleex}{
\begin{tikzpicture}
  \exphdef
\end{tikzpicture}
}\end{center}
}

% 2 : Processus
\only<2>{
\tikzstyle{process}=[circle,draw,minimum size=15pt,font=\footnotesize,inner sep=1pt]
\tikzstyle{tick label}=[font=\footnotesize]
\tikzstyle{tick}=[densely dotted]
\tikzstyle{hit}=[transparent]
\tikzstyle{selfhit}=[transparent, min distance=30pt,curve to]
\tikzstyle{bounce}=[transparent]
\tikzstyle{hlhit}=[transparent]
\begin{center}\scalebox{\scaleex}{
\begin{tikzpicture}
  \exphdef
\end{tikzpicture}
}\end{center}
}

% 3 : États
\only<3>{
\tikzstyle{hit}=[transparent]
\tikzstyle{selfhit}=[transparent, min distance=30pt,curve to]
\tikzstyle{bounce}=[transparent]
\tikzstyle{hlhit}=[transparent]
\begin{center}\scalebox{\scaleex}{
\begin{tikzpicture}
  \exphdef

  \TState{3}{a_0,b_1,z_0}
\end{tikzpicture}
}\end{center}
}

% 4 : Actions
\only<4->{
\tikzstyle{tick}=[densely dotted]
\tikzstyle{hit}=[->,>=angle 45]
\tikzstyle{selfhit}=[min distance=30pt,curve to]
\tikzstyle{bounce}=[densely dotted,>=stealth',->]
\tikzstyle{hlhit}=[very thick]
\begin{center}\scalebox{\scaleex}{
\begin{tikzpicture}
  \exphdef

  \TState{4-5}{a_0,b_1,z_0}
  \TState{6}{a_0,b_1,z_1}
  \TState{7}{a_1,b_1,z_1}
  \TState{8}{a_1,b_1,z_2}

  \only<5>{
    \THit{b_1}{hl}{z_0}{.west}{z_1}
    \path[bounce,bend left,hl] \TBounce{z_0}{}{z_1}{.south};
  }
  \only<6>{
    \THit{a_0}{out=250,in=200,selfhit,hl}{a_0}{.west}{a_1}
    \path[bounce,bend left,hl] \TBounce{a_0}{}{a_1}{.south};
  }
  \only<7>{
    \THit{a_1}{hl}{z_1}{.west}{z_2}
    \path[bounce,bend left,hl] \TBounce{z_1}{}{z_2}{.south};
  }
\end{tikzpicture}
}\end{center}
}

%\medskip
\begin{liste}
  \item \tval{Sorts}: components \qex{$a$, $b$, $z$}
\pause[2]
  \item \tval{Processes}: local states / levels of expression \qex{$z_0$, $z_1$, $z_2$}
\pause[3]
  \item \tval{States}: sets of active processes%
  \only<3-5>{\qex{$\PHetat{a_0, b_1, z_0}$}}%
  \only<6>{\qex{$\PHetat{a_0, b_1, z_1}$}}%
  \only<7>{\qex{$\PHetat{a_1, b_1, z_1}$}}%
  \only<8>{\qex{$\PHetat{a_1, b_1, z_2}$}}%
\pause[4]
  \item \tval{Actions}: dynamics \qex{\only<5>{\underline}{$\PHfrappe{b_1}{z_0}{z_1}$}, \only<6>{\underline}{$\PHfrappe{a_0}{a_0}{a_1}$}, \only<7>{\underline}{$\PHfrappe{a_1}{z_1}{z_2}$}}
\end{liste}
\end{frame}



\subsection{Static analysis}

\begin{frame}[t]
  \frametitle{Static analysis: successive reachability}
  \framesubtitle{\tcite{\citepmrmscs}}

%Successive reachability of processes:

\begin{columns}
\begin{column}{0.55\textwidth}

\begin{center}
\scalebox{0.75}{
\begin{tikzpicture}
  \path[use as bounding box] (-1,-3) rectangle (7,2.7);
  \exatt

  \TState{1-4}{a_0,b_0,c_0,d_0}

  \TState{5}{a_0,b_0,c_0,d_0}
  \TState{6}{a_0,b_0,c_1,d_0}
  \TState{7}{a_0,b_0,c_1,d_1}
  \TState{8}{a_0,b_1,c_1,d_1}
  \TState{9}{a_0,b_1,c_1,d_2}

  \node<2>[objective] at (d_1.center) {1?};
  \node<2>[objective] at (d_2.center) {2?};

  \node<3>[objective] at (d_1.center) {1?};
  \node<3>[objective] at (b_1.center) {2?};
  \node<3>[objective] at (d_2.center) {3?};

  \node<4-8>[objective] at (d_2.center) {1?};
  %\node<9>[process,very thick,fill=none] at (d_2.center) {};

  \only<5>{\THit{a_0}{hlhit}{c_0}{.north}{c_1}}
  \path<5>[bounce,bend left,hlhit] \TBounce{c_0}{}{c_1}{.west};
  \only<6>{\THit{b_0}{hlhit}{d_0}{.west}{d_1}}
  \path<6>[bounce,bend left,hlhit] \TBounce{d_0}{}{d_1}{.south};
  \only<7>{\THit{c_1}{bend left=20pt,hlhit}{b_0}{.west}{b_1}}
  \path<7>[bounce,bend left,hlhit] \TBounce{b_0}{}{b_1}{.south};
  \only<8>{\THit{b_1}{hlhit}{d_1}{.west}{d_2}}
  \path<8>[bounce,bend left,hlhit] \TBounce{d_1}{}{d_2}{.south};
\end{tikzpicture}
}
\end{center}

\end{column}
\begin{column}{0.45\textwidth}

%\pause
~\\~\\
\begin{itemize}
  \item Initial state
    \\ \rex{\PHetat{a_1, b_0, c_0, d_0}} \pause
  \item Objectives
    \\ \rex{$[\ \Rsh d_1\ \PHconcat\ \Rsh d_2\ ]$} \pause
    \\\smallskip \rex{$[\ \Rsh d_1 \PHconcat\ \Rsh b_1 \PHconcat\ \Rsh d_2\ ]$} \pause
    \\\smallskip \rex{$[\ \Rsh d_2\ ]$} \pause
\end{itemize}

\end{column}
\end{columns}

\medskip
\begin{center}
\f Concretization of the objective = scenario

\ex{
$ \only<5>{\underline{\PHfrappe{a_0}{c_0}{c_1}}} \only<-4,6->{\PHfrappe{a_0}{c_0}{c_1}} \PHconcat %
  \only<6>{\underline{\PHfrappe{b_0}{d_0}{d_1}}}\only<-5,7->{\PHfrappe{b_0}{d_0}{d_1}} \PHconcat %
  \only<7>{\underline{\PHfrappe{c_1}{b_0}{b_1}}}\only<-6,8->{\PHfrappe{c_1}{b_0}{b_1}} \PHconcat %
  \only<8>{\underline{\PHfrappe{b_1}{d_1}{d_2}}}\only<-7,9->{\PHfrappe{b_1}{d_1}{d_2}}
$}
\end{center}
\end{frame}



\begin{frame}
  \frametitle{Over- and Under-approximations}
  \framesubtitle{\tcite{\citepmrmscs}}

\begin{fleches}
  \item Directly checking $R$ is hard (\tval{exponential})
  \item Rather check \tval{approximations} $P$ and $Q$ so that: $\underline{P \Rightarrow R \Rightarrow Q}$
\end{fleches}

\begin{center}
\scalebox{0.7}{
  \figsa
}
\end{center}

\uncover<8->{
%Polynomial w.r.t.~the number of sorts and \\exponential w.r.t.~the number of processes in each sort
Computing $P$ or $Q$ is much simpler (roughly \tval{polynomial})
\begin{fleches}
  \item Efficient for big models \f \tval{Hundredths of seconds}
\end{fleches}
}
\end{frame}



\begin{frame}[t]
  \frametitle{Under-approximation}

\def \tu {3}
\def \tub {4}
\def \tuf {5}

\begin{columns}[t]
\begin{column}{0.48\textwidth}
\begin{center}
\scalebox{0.55}{
\begin{tikzpicture}
  \path[use as bounding box] (-1,-2.7) rectangle (7,1.5);
  \exatt
  \TState{-\tu}{a_1,b_1,c_1,d_0}
  \TState{\tub-}{a_0,b_1,c_0,d_0}
  \node[objective] (d_2) at (d_2.center) {?};
\end{tikzpicture}
}
\end{center}

\end{column}
\begin{column}{0.52\textwidth}

\uncover<2->{
%\vspace{1em}
\tval{Sufficient condition}:

\smallskip
\begin{itemize}
  \item no cycle
  \item \only<-\tu>{each objective has a solution} \only<\tub->{\sout{each objective has a solution}}
\end{itemize}
\begin{center}\Large%
  \only<\tu>{\textcolor{darkgreen}{$P$ is \textbf{true} $\Rightarrow$ $R$ is \textbf{true}}}%
  \only<\tuf>{\textcolor{darkyellow}{$P$ is \textbf{false} $\Rightarrow$ \textbf{Inconclusive}}}
\end{center}
}

\end{column}
\end{columns}

\begin{center}%
%\vspace*{1cm}%
\scalebox{\scaleex}{%
\only<-\tu>{%
\scalebox{\scaleex}{%
\begin{tikzpicture}[aS]
  \path[use as bounding box] (.7,1) rectangle (5.8,2.5);

  \glclegend{}{$d_2$}{$\PHobj{d_0}{d_2}$}
\end{tikzpicture}
}
~~~\sauyes
}
\only<\tub->{
  \sauinconc
}}
\end{center}
\end{frame}



\begin{frame}[t]
  \frametitle{Over-approximation}

\def \to {4}
\def \tob {5}
\def \tof {6}
\def \tokp {7}

\begin{columns}[t]
\begin{column}{0.48\textwidth}
\begin{center}
\scalebox{0.55}{
\begin{tikzpicture}
  \path[use as bounding box] (-1,-3.5) rectangle (7,1.5);
  \exatt
  \TState{-\to}{a_1,b_0,c_0,d_1}
  \TState{\tob-}{a_1,b_1,c_1,d_0}
  \node[objective] (d_2) at (d_2.center) {?};
\end{tikzpicture}
}
\end{center}
\bigskip

\end{column}
\begin{column}{0.52\textwidth}

\tval{Necessary condition}:

\smallskip
\only<2->{
\only<3-\to>{\sout{There exists a traversal}}\only<2,\tob->{There exists a traversal}
with no cycle

\smallskip
\begin{itemize}
  \item \only<3-\to>{\sout{objective $\rightarrow$ follow \tval{one} solution}}\only<1-2,\tob->{objective $\rightarrow$ follow \tval{one} solution}%
  \item solution $\rightarrow$ follow \tval{all} processes
  \item process $\rightarrow$ follow \tval{all} objectives
\end{itemize}
\begin{center}\Large%
  \only<\to>{\textcolor{red}{$Q$ is \textbf{false} $\Rightarrow$ $R$ is \textbf{false}}}%
  \only<\tof->{\textcolor{darkyellow}{\textbf{$R$ is \textbf{true} $\Rightarrow$ Inconclusive}}}
\end{center}
}

\end{column}
\end{columns}

%\bigskip

\begin{center}
\scalebox{\scaleex}{
\only<1-\to>{
  \saono
}
\only<\tob->{
  \saoinconc
}}
\end{center}
\end{frame}

% Contributions

\subsection{Personal contributions}

\begin{frame}[c]
  \frametitle{Translation of PH models}
  \framesubtitle{\tcite{\citefpimrcmsb}}

\begin{columns}
\begin{column}{0.5\textwidth}

\begin{flushleft}
\vspace*{1cm}
\hspace*{0.5cm}
\scalebox{\scaleinf}{
\begin{tikzpicture}
  \path[use as bounding box] (-0.5,2) rectangle (6.5,4);
  \path[draw,rounded corners] (-1.5,-1) rectangle (7.5,5);
  \exphinfblack{}
\end{tikzpicture}
}
\end{flushleft}
\vspace{1.3cm}
\begin{center}
  \tval{Process Hitting}
  
  Efficient but recent
\end{center}

\end{column}
\begin{column}{0.15\textwidth}

\vspace*{-1.5cm}
\begin{center}
\begin{tikzpicture}
  \path[use as bounding box] (1,-0.25) rectangle (2,0.25);
  \path[draw=couleurtheme]<3-> (0.6, -0.3) edge[<-,draw,very thick,dashed] (2.6, -0.3);
  \path[draw=couleurtheme]<4> (0.6, -0.7) edge[->,draw,very thick] (2.6, -0.7);
\end{tikzpicture}
\end{center}

\end{column}
\begin{column}{0.35\textwidth}

\uncover<2->{
\vspace*{1.4cm}
\begin{flushright}
\begin{tikzpicture}[adn]
  \path[use as bounding box] (.5,-1) rectangle (2,1.5);
  \path[draw,rounded corners] (-0.6,-.9) rectangle (2.6,2.3);
  % Gènes
  \node[inner sep=0] (a) at (0,1.5) {a};
  \node[inner sep=0] (b) at (0,0) {b};
  \node[inner sep=0] (z) at (2,0.75) {z};
  % Arcs
  \path (a) edge[act] node[elabel,above=-2pt] {$+$} (z);
  \path (b) edge[inh] node[elabel,below=-2pt] {$-$} (z);
\end{tikzpicture}
\hspace*{0.9cm}
\end{flushright}
\smallskip
\vspace*{-.5cm}
\begin{center}
  \tval{Usual modelling}\hspace*{0.4cm}
  
  Widespread \& readable\hspace*{0.4cm}
\end{center}
}

\end{column}
\end{columns}

\end{frame}



\newcommand{\bigneq}[2]{\tikz{\path[use as bounding box] (-.5,-.75) rectangle (.5,.5);\node[thick,draw=notsodarkred!#1,circle,minimum size=1cm,inner sep=0,fill=notsodarkred!#2,text=notsodarkred!#1] {\Huge$\neq$};}}
\newcommand{\bigeq}{\tikz{\path[use as bounding box] (-.5,-.75) rectangle (.5,1);\node[thick,draw=darkgreen,circle,inner sep=0,minimum size=1cm,text height=.5cm,fill=darkgreen!20,text=darkgreen] {\Huge$=$};}}

\begin{frame}[c]
  \frametitle{Enrich PH semantics}
  \framesubtitle{\tcite{\citefpmrcstobio}}

\begin{columns}
\begin{column}{0.5\textwidth}

\begin{flushleft}
\vspace*{1cm}
\hspace*{0.5cm}
\only<1>{\scalebox{\scaleinf}{
\begin{tikzpicture}
  \path[use as bounding box] (-0.5,2) rectangle (6.5,4);
  \path[draw,rounded corners] (-1.5,-1) rectangle (7.5,5);
  \exphinfblack{}
\end{tikzpicture}
}}%
\only<2->{\scalebox{\scaleinf}{
\begin{tikzpicture}
  \path[use as bounding box] (-0.5,2) rectangle (6.5,4);
  \path[draw,rounded corners] (-1.5,-1) rectangle (7.5,5);
  \exphinfblack{prio}
\end{tikzpicture}
}}
\end{flushleft}
\vspace{1.3cm}
\begin{center}
  \tval{Process Hitting}
  
  \only<1>{Loose behaviour}
  \only<2->{Accurate behaviour}
\end{center}

\end{column}
\begin{column}{0.15\textwidth}

\vspace*{-1.5cm}
\begin{center}
\vspace*{1cm}
\only<1>{\bigneq{100}{30}}%
\only<2->{\bigneq{30}{10}}

\begin{tikzpicture}
  \path[use as bounding box] (1,-0.25) rectangle (2,0.25);
  \path[draw=couleurtheme]<2-> (1.5, -0.75) edge[<-,draw,very thick] (1.5, .5);
\end{tikzpicture}

\uncover<2->{\bigeq}
\end{center}

\end{column}
\begin{column}{0.35\textwidth}

\vspace*{1.4cm}
\begin{flushright}
\begin{tikzpicture}[adn]
  \path[use as bounding box] (0,-1) rectangle (2,1.5);
  \path[draw,rounded corners] (-0.6,-.9) rectangle (2.6,2.3);
  %\path[draw] (0,-0.7) rectangle (2,-0.7);
  % Gènes
  \node[inner sep=0] (a) at (0,1.5) {a};
  \node[inner sep=0] (b) at (0,0) {b};
  \node[inner sep=0] (z) at (2,0.75) {z};
  % Arcs
  \path (a) edge[act] node[elabel,above=-2pt] {$+$} (z);
  \path (b) edge[inh] node[elabel,below=-2pt] {$-$} (z);
\end{tikzpicture}
\hspace*{0.9cm}
\end{flushright}
\smallskip
\vspace*{-.5cm}
\begin{center}
  \tval{Usual modelling}\hspace*{0.4cm}
  
  ~Accurate behaviour\phantom{j}\hspace*{0.4cm}
\end{center}

\end{column}
\end{columns}

\end{frame}


\section{What for?}
% What for

\begin{frame}
  \frametitle{What for?}

\begin{center}
\LARGE
  Very well, but…

\Huge
  \vspace*{1em}
  \pause
  \tval{What's the point?}
  \vspace*{.2em}
\end{center}

\pause
\begin{columns}
\begin{column}{.5\textwidth}

\begin{itemize}
  \item Validating the models
  \item Predicting behaviours
  \item Finding gene therapies
\end{itemize}

\end{column}
\end{columns}

\end{frame}



\tikzstyle{abox}=[rounded corners, very thick, align=center]
\tikzstyle{anedge}=[draw=black, ->, shorten <=5pt, shorten >=5pt, very thick]
\tikzstyle{bigbox}=[rounded corners, line width=3pt]
\definecolor{couleurcellule}{rgb}{1,.5,.3}

\begin{frame}
  \frametitle{Experiments \textit{in silico}}

\begin{center}
\scalebox{1.3}{
\begin{tikzpicture}
  \path[use as bounding box] (-1,-3.5) rectangle (5,1);
  % Rectangles labos
  \path[bigbox, draw=blue!50, fill=blue!10] (-2,-1.2) rectangle (6,1.2);
  \node[darkblue!50, anchor=north east] at (6,1.2) {\textsc{wet lab}};
  \path[bigbox, draw=yellow!80, fill=yellow!20] (-2,-1.9) rectangle (6,-4.1);
  \node[darkyellow!50, anchor=north east] at (6,-1.9) {\textsc{dry lab}};
  % Nœuds
  \node[ellipse, draw=pink!100, fill=pink!30, decoration={coil}, decorate, align=center, very thick, inner sep=6.7pt] (sb) {Biological\\system};
  \node[ellipse, draw=violet!80, fill=violet!15, align=center, thick, inner sep=2pt] at (sb) (system) {Biological\\system};
  \node<2->[abox, draw=darkblue, fill=darkblue!30, right of=system, node distance=4cm] (exp) {Experiments\\\textit{in vivo / ex vivo}};
  \node<3-4>[abox, ellipse, draw=notsodarkred, fill=red!30, below of=exp, node distance=3cm] (model) {Model};
  \node<5->[abox, ellipse, draw=notsodarkgreen, fill=green!30, below of=exp, node distance=3cm] (model) {Model};
  \node<4-5>[abox, draw=darkyellow, fill=yellow!40, left of=model, node distance=4cm] (valid) {Validation\\\textit{in silico}};
  \node<7->[abox, draw=darkyellow, fill=yellow!40, draw=darkblue, fill=darkblue!30] at (valid) (predict) {Experiments\\\textit{in silico}};
  % Arcs
  \path<2->[anedge, bend left, shorten <=8pt, shorten >=8pt] (sb) edge (exp);
  \path<3->[anedge] (exp) edge (model);
  \path<4-5,7->[anedge, bend left] (model) edge (valid);
  \path<5>[anedge, bend left] (valid) edge (model.north west);
  \path<5>[anedge, bend right, in=240] (exp) edge (model.north west);
  \path<8>[anedge] (predict) edge (exp.south west);
  \path<8>[anedge, bend left, in=110] (model) edge (exp.south west);
\end{tikzpicture}}
\end{center}
\end{frame}



\tikzstyle{active}=[fill=yellow!70, thick]
\tikzstyle{inactive}=[draw=gray, thick]
\tikzstyle{badgene}=[active, draw=notsodarkred, -, decorate,decoration={zigzag,segment length=1pt,amplitude=.5pt}]
\tikzstyle{inactivebadgene}=[inactive, -, decorate,decoration={zigzag,segment length=1pt,amplitude=.5pt}]

\begin{frame}
  \frametitle{Gene therapies}

\tval{Modify} DNA to cure a disease

\begin{itemize}
  \item Replace a mutated gene \f remove a \tval{harmful protein}
  \item Add a new gene \f produce a \tval{therapeutic protein}
\end{itemize}

\medskip
\begin{center}
\scalebox{1}{
\begin{tikzpicture}[adn]
  %\path[use as bounding box] (-0.7,-0.4) rectangle (2.5,2);
  \node[active] (a) at (-1,0) {a}; \node[active] (b) at (0,0) {b}; \node[active] (c) at (1,0) {c};
  \node[active] (d) at (-.5,-1) {}; \node[active] (e) at (.5,-1) {}; \node[active] (f) at (1.5,-1) {};
  \node[active] (g) at (-1,-2) {}; \node<1>[active] (h) at (0,-2) {n};
  \node[active] (i) at (-1.5,-3) {}; \node<-2>[active] (j) at (-.5,-3) {}; \node<-2>[active] (k) at (.5,-3) {};
  \node<-2>[badgene] (x) at (-1,-4) {x}; \node<-2>[badgene] (y) at (0,-4) {y}; \node<-2>[badgene] (z) at (1,-4) {z};

  \node<2->[inactive] at (h) {n};
  \node<3->[inactive] at (j) {}; \node<3->[inactive] at (k) {};
  \node<3->[inactivebadgene] at (x) {x}; \node<3->[inactivebadgene] at (y) {y}; \node<3->[inactivebadgene] at (z) {z};

%  \uncover<2->{\path[draw=cyan, fill=gray!50, -, thick] (.4,-2) -- (1.5,-2) -- (1.2,-1.8) -- (2.5,-1.8) -- (2.9,-2) -- (1.7,-2) -- (2,-2.2) -- (.4,-2);}
  \uncover<2->{\path[shading=1, left color=notsodarkred, right color=red!30]
    (.4,-2) -- (1.5,-2.1) -- (1.2,-1.9) -- (2.5,-2.05) -- (2.8,-2.25) -- (1.7,-2.1) -- (2,-2.3) -- (.4,-2);}

  \path
    (a) edge[act] (d) (b) edge[act] (d) (b) edge[inh] (e) (c) edge[act] (e) (c) edge[act] (f)
    (d) edge[inh, bend left] (g) (d) edge[inh] (h) (e) edge[act] (h)
    (g) edge[act, bend left] (i) (g) edge[act, bend left] (d) (h) edge[act] (k) (h) edge[act, bend left] (j)
    (i) edge[act, bend left] (g) (j) edge[act] (x) (j) edge[act] (y) (k) edge[act] (z) (j) edge[inh, bend left] (h);
\end{tikzpicture}}
\end{center}
\end{frame}


\section{Summary \& Conclusion}
% Performances et conclusion

\begin{frame}[c]
  \frametitle{Summary \& Conclusion}

\begin{itemize}
  \item What is Bio-informatics?
  \begin{fleches}
    \item Qualitative modelling of \tval{gene regulations}
    \item Large models are hard to study (\tval{exponential})
  \end{fleches}
  
  \smallskip
  \item What do I do?
  \begin{fleches}
    \item The \tval{Process Hitting} modelling
    \item Very efficient on large-scale models (\tval{polynomial})
    \item 2 important contributions
  \end{fleches}
  
  \smallskip
  \item What for?
  \begin{fleches}
    \item \tval{Validating} \& \tval{utilizing} biological models
    \item Gene therapies
  \end{fleches}
\end{itemize}
\end{frame}


\appendix
\section[x]{Bibliography}
% Bibliographie

\begin{frame}[c]
  \frametitle{Bibliography}

\footnotesize
\setlength{\parindent}{-1em}
\setlength{\parskip}{0.5em}
~

\vfill

\tcitebullet Loïc Paulevé, Morgan Magnin, Olivier Roux. \tscite{Refining dynamics of gene regulatory networks in a stochastic $\pi$-calculus framework}. In Corrado Priami, Ralph-Johan Back, Ion Petre, and Erik de Vink, editors: \textit{Transactions on Computational Systems Biology XIII}, Lecture Notes in Computer Science, 171-191. Springer Berlin Heidelberg, 2011.

\tcitebullet Loïc Paulevé, Morgan Magnin, Olivier Roux. \tscite{Static analysis of biological regulatory networks dynamics using abstract interpretation}. \textit{Mathematical Structures in Computer Science}. 2012.

%\tcite{RCB08} Adrien Richard, Jean-Paul Comet, Gilles Bernot. \tscite{R. Thomas' logical method}, 2008. Invited at \textit{Tutorials on modelling methods and tools: Modelling a genetic switch and Metabolic Networks}, Spring School on Modelling Complex Biological Systems in the Context of Genomics.

%\tcite{RCB06} Adrien Richard, Jean-Paul Comet, Gilles Bernot. \textit{Modern Formal Methods and App.}, chapter \tscite{Formal Methods for Modeling Biological Regulatory Networks}, 83--122. 2006.

\tcitebullet Hidde de Jong. \tscite{Modeling and simulation of genetic regulatory systems: a literature review}, \textit{Journal of Computational biology} 9(1), 67--103. 2002.

%\tcitebullet Adrien Richard and Jean-Paul Comet. \tscite{Necessary conditions for multistationarity in discrete dynamical systems}. \textit{Discrete Applied Mathematics} 155(18), 2403--2413. 2007.

%\tcitebullet Élisabeth Remy, Paul Ruet and Denis Thieffry. \tscite{Graphic requirements for multistability and attractive cycles in a boolean dynamical framework}, \textit{Advances in Applied Mathematics} 41(3), 335-350. Elsevier, 2008.

%\tcitebullet Gilles Bernot, Jean-Paul Comet, Adrien Richard and Janine Guespin. \tscite{Application of formal methods to biological regulatory networks: extending Thomas' asynchronous logical approach with temporal logic}, \textit{Journal of Theoretical Biology}, 229(3), 339--347. Elsevier, 2004.

%\tcitebullet Sohei Ito, Naoko Izumi, Shigeki Hagihara and Naoki Yonezaki. \tscite{Qualitative analysis of gene regulatory networks by satisfiability checking of Linear Temporal Logic}, in 2010 IEEE International Conference on \textit{BioInformatics and BioEngineering} (BIBE), 232--237. IEEE, 2010.

\tcitebullet Maxime Folschette, Loïc Paulevé, Katsumi Inoue, Morgan Magnin, Olivier Roux. \tscite{Concretizing the Process Hitting into Biological Regulatory Networks}. In David Gilbert and Monika Heiner, editors, \textit{Computational Methods in Systems Biology X}, Lecture Notes in Computer Science, 166--186. Springer Berlin Heidelberg, 2012.

%\tcite{Paulevé11} Loïc Paulevé. PhD thesis: \tscite{\textit{Modélisation, Simulation et Vérification des Grands Réseaux de Régulation Biologique}}, October 2011, Nantes, France

%\tcite{PMR10-TSE} Loïc Paulevé, Morgan Magnin, and Olivier Roux. \textit{Tuning Temporal Features within the Stochastic $\pi$-Calculus}. IEEE Transactions on Software Engineering, 37(6):858-871, 2011.

%\tcite{PR10-CRAS} Loïc Paulevé and Adrien Richard. \textit{Topological Fixed Points in Boolean Networks}. Comptes Rendus de l'Académie des Sciences - Series I - Mathematics, 348(15-16):825 - 828, 2010.

\vfill
\Large
\begin{flushright}
  \tval{Thank you}\hspace{1cm}~
\end{flushright}
\vfill

~

\end{frame}


\end{document}
